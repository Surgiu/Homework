\documentclass[12pt,a4paper]{article}

\usepackage{amsmath}
\usepackage{amssymb}
\usepackage{graphicx}
\usepackage{CJKutf8}

\graphicspath{images/} %TODO

\title{Homework 1} %TODO
\author{Qijun Han 12212635}
\date{\today}


\begin{document}
\maketitle

\section{NE1.1}
\subsection{Problem a}

\[G(s) = \frac{1}{s^2+2s+6}\]

ODE:
\[y''(t) + 2y'(t) + 6y(t) = u(t)\]

let $x_1(t) = y(t), x_2(t) = y'(t)$, then we have

$\begin{cases}
    x_1'(t) = x_2(t) \\
    x_2'(t) = -2x_2(t) - 6x_1(t) + u(t)
\end{cases}
$

so the state space representation is
\begin{equation}
    \begin{aligned}
        \begin{bmatrix}
            {x_1(t)} \\
            {x_2(t)}
        \end{bmatrix}' & = \begin{bmatrix}
                               0  & 1  \\
                               -6 & -2
                           \end{bmatrix} \begin{bmatrix}
                                             x_1(t) \\
                                             x_2(t)
                                         \end{bmatrix} + \begin{bmatrix}
                                                             0 \\
                                                             1
                                                         \end{bmatrix} u(t) \\
        y(t)               & = \begin{bmatrix}
                                1 & 0
                            \end{bmatrix} \begin{bmatrix}
                                              x_1(t) \\
                                              x_2(t)
                                          \end{bmatrix}
    \end{aligned}
\end{equation}

thus
$
    A =
    \begin{bmatrix}
        0  & 1  \\
        -6 & -2
    \end{bmatrix},
    B =
    \begin{bmatrix}
        0 \\
        1
    \end{bmatrix},
    C = \begin{bmatrix}
        1 & 0
    \end{bmatrix},
    D =
    \begin{bmatrix}
        0
    \end{bmatrix}
$

\subsection{Problem b}
let $G_1(s)=\frac{1}{s^2+2s+6}=\frac{W(s)}{U(s)}$, $G_2(s)=s+3=\frac{Y(s)}{W(s)}$, then
\[G(s)  = G_1(s)G_2(s) \]
ODE:
\begin{equation}
    \begin{aligned}
        w''(t) + w'(t) + w(t) & = u(t)    \\
        y(t)            & = w'(t) + 3w(t)
    \end{aligned}
\end{equation}
let $ x_1(t) = w(t), x_2(t) = w'(t)$, then we have
\begin{equation}
    \begin{aligned}
        \begin{bmatrix}
            {x_1(t)} \\
            {x_2(t)}
        \end{bmatrix}' & = \begin{bmatrix}
                               0  & 1  \\
                               -6 & -2
                           \end{bmatrix} \begin{bmatrix}
                                             x_1 \\
                                             x_2
                                         \end{bmatrix} + \begin{bmatrix}
                                                             0 \\
                                                             1
                                                         \end{bmatrix} u(t) \\
        y               & = \begin{bmatrix}
                                1 & 0
                            \end{bmatrix} \begin{bmatrix}
                                              x_1(t) \\
                                              x_2(t)
                                          \end{bmatrix}
    \end{aligned}
\end{equation}
and
\[ y(t) = \begin{bmatrix}
        3 & 1
    \end{bmatrix} \begin{bmatrix}
                      x_1(t) \\
                      x_2(t)
                  \end{bmatrix} \]

thus
$
    A =
    \begin{bmatrix}
        0  & 1  \\
        -6 & -2
    \end{bmatrix},
    B =
    \begin{bmatrix}
        0 \\
        1
    \end{bmatrix},
    C = \begin{bmatrix}
        3 & 1
    \end{bmatrix},
    D =
    \begin{bmatrix}
        0
    \end{bmatrix}
$

\subsection{Problem c}
similar to (a), we have
$
A = \begin{bmatrix}
    0  & 1 & 0  \\
    0 & 0 & 1 \\
    -6 & -8 & -4
\end{bmatrix},
B = \begin{bmatrix}
    0 \\
    0 \\
    10
\end{bmatrix},
C = \begin{bmatrix}
    1 & 0 &0
\end{bmatrix},
D = \begin{bmatrix}
    0
\end{bmatrix}
$

\subsection{Problem d}
similar to (b), we have
$
A = \begin{bmatrix}
    0 & 1 & 0 & 0 \\
    0 & 0 & 1 & 0 \\
    0 & 0 & 0 & 1 \\
    -66 & -44 & -11 & -10 \\
\end{bmatrix},
B= \begin{bmatrix}
    0 &0 &0 &1
\end{bmatrix},
C= \begin{bmatrix}
    6 & 4 & 1 & 0
\end{bmatrix},
D = \begin{bmatrix}
    0
\end{bmatrix}
$

\section{AE1.11}

suppose A is an $m \times m$ matrix, B is an $n \times n$ matrix.
to prove
\[
\begin{bmatrix}
    A & D \\
    C & B
\end{bmatrix}^{-1} = \begin{bmatrix}
    A^{-1}+E\Delta^{-1}F & -E\Delta^{-1} \\
    -\Delta^{-1}F & \Delta^{-1}
\end{bmatrix}
\], we need to prove that
\[
    \begin{bmatrix}
        A & D \\
        C & B
    \end{bmatrix} \begin{bmatrix}
        A^{-1}+E\Delta^{-1}F & -E\Delta^{-1} \\
        -\Delta^{-1}F & \Delta^{-1}
    \end{bmatrix} = \begin{bmatrix}
        I_m & 0 \\
        0 & I_n
    \end{bmatrix}
\]

first, consider the left-top block:
\[
    \begin{aligned}
        A(A^{-1}+E\Delta^{-1}F) + D(-\Delta^{-1}F) & = AA^{-1} + AE\Delta^{-1}F - D\Delta^{-1}F \\
                                                     & = I_m + AE\Delta^{-1}F - D\Delta^{-1}F
    \end{aligned}
\]
since $E = A^{-1}D$, we have $AE\Delta^{-1}F = D\Delta^{-1}F$, thus we get $I_m$.

second, consider the right-top block:
\[
    \begin{aligned}
        A(-E\Delta^{-1}) + D(\Delta^{-1}) & = -AE\Delta^{-1} + D\Delta^{-1} \\
                                          & = -D\Delta^{-1} + D\Delta^{-1} \\
                                          & = 0
    \end{aligned}
\]

third, consider the left-bottom block:
\[
    \begin{aligned}
        C(A^{-1}+E\Delta^{-1}F) + B(-\Delta^{-1}F) & = CA^{-1} + CE\Delta^{-1}F - B\Delta^{-1}F \\
                                                & = CA^{-1} + (CE - B)\Delta^{-1}F \\
                                                & = CA^{-1} + (B - CA^{-1}D)\Delta^{-1}F \\
                                                & = CA^{-1} - \Delta \Delta^{-1} F \\
                                                & = CA^{-1} - F \\
                                                & = 0
    \end{aligned}
\]
since $E = A^{-1}D$, we have $CE\Delta^{-1}F = B\Delta^{-1}F$, thus we get $0$.

fourth, consider the right-bottom block:
\[
    \begin{aligned}
        C(-E\Delta^{-1}) + B(\Delta^{-1}) & = -CA^{-1}D\Delta^{-1} + B\Delta^{-1} \\
                                          & = (B - CA^{-1}D)\Delta^{-1} \\
                                          & = \Delta^{-1}\Delta \\
                                        & = I_n
    \end{aligned}
\]

thus we have proved that
\[
    \begin{bmatrix}
        A & D \\
        C & B
    \end{bmatrix}^{-1} = \begin{bmatrix}
        A^{-1}+E\Delta^{-1}F & -E\Delta^{-1} \\
        -\Delta^{-1}F & \Delta^{-1}
    \end{bmatrix}
\]
Q.E.D.

\section{AE1.12}
notice that the Jordan block is an upper triangular matrix, 
thus the inverse of a Jordan block is also an upper triangular matrix.

so we suppose the inverse of the k-th Jordan block has the form of:
\[
    \begin{bmatrix}
        a_{11} & a_{12} & \cdots & a_{1k} \\
        0      & a_{22} & \cdots & a_{2k} \\
        \vdots & \vdots & \ddots & \vdots \\
        0      & 0      & \cdots & a_{kk}

    \end{bmatrix}
\]
from the fact that the multiplication of a Jordan block and its inverse is an identity matrix, we have
\[
    \begin{bmatrix}
        \lambda & 1      & 0      & \cdots & 0      \\
        0       & \lambda & 1      & \cdots & 0      \\
        0       & 0      & \lambda & \cdots & 0      \\
        \vdots  & \vdots & \vdots & \ddots & \vdots \\
        0       & 0      & 0      & \cdots & \lambda
    \end{bmatrix} \begin{bmatrix}
        a_{11} & a_{12} & \cdots & a_{1k} \\
        0      & a_{22} & \cdots & a_{2k} \\
        \vdots & \vdots & \ddots & \vdots \\
        0      & 0      & \cdots & a_{kk}
    \end{bmatrix} = \begin{bmatrix}
        1 & 0 & \cdots & 0 \\
        0 & 1 & \cdots & 0 \\
        \vdots & \vdots & \ddots & \vdots \\
        0 & 0 & \cdots & 1
    \end{bmatrix}
\]
thus we have
\[
    \begin{aligned}
        a_{11} \lambda & = 1 \\
        a_{12} \lambda & = 0 \\
        a_{22} \lambda & = 1 \\
        a_{13} \lambda & = 0 \\
        a_{23} \lambda & = 0 \\
        a_{33} \lambda & = 1 \\
        \vdots         & = \vdots \\
        a_{1k} \lambda & = 0 \\
        a_{2k} \lambda & = 0 \\
        \vdots         & = \vdots \\
        a_{kk} \lambda & = 1
    \end{aligned}
\]

thus the inverse of the k-th Jordan block is
\[
    \begin{bmatrix}
        \frac{1}{\lambda} &- \frac{1}{\lambda^2} & -\frac{1}{\lambda^2} & \cdots & (-1)^{k-1} \frac{1}{\lambda^{k-1}} \\
        0                 & \frac{1}{\lambda}     & -\frac{1}{\lambda^2} & \cdots & (-1)^{k-2} \frac{1}{\lambda^{k-1}} \\
        0                 & 0                     & \frac{1}{\lambda}     & \cdots & (-1)^{k-3} \frac{1}{\lambda^{k-1}} \\
        \vdots            & \vdots                & \vdots                & \ddots & \vdots \\
        0                 & 0                     & 0                     & \cdots & \frac{1}{\lambda}
    \end{bmatrix}
\]

\section{Problem 4}

\end{document}